%!TEX root = ../thesis.tex

\chapter{Мінімальна задача Комівояжера}
\label{chap:review}

\section{Формулювання}

\emph{ Задача Комівояжера(TSP, нім. Problem des Handlungsreisenden) }
Нехай задано $n$ міст та ціле невідємне число $d_{ij}$ між будь-якими
двома містами $i$ та $j$ ( припустимо, що індекси симетричні $d_{ij} = d_{ji}$)
Потрібно знайти \emph{найкоротший шлях} між містами такий, щоб перестановка 
$\pi$ на множині $[1...n]$ така, що $\sum\limits_{i = 0}^{n}{d_{\pi(i)\pi(i+1)}}$ ( де $\pi(n+1) = \pi(n)$ )
була найменшою.

Власне, обрана модифікація задачі Комівояжера: 
\\ \emph{Мінімальна задача Комівояжера: } дано $n$ міст, дистанції між кожними містами і
маршрут $T$, і потрібно відповісти, чи ми проходимо по кожному місту в машруті T рівно один раз
та чи $T$ мінімальної довжини?

\section{Історичний екскурс}

Невідомо, коли проблему комівояжера було досліджено вперше. Однак, відома видана в 1832 році книжка з назвою
«Комівояжер — як він має поводитись і що має робити для того, аби доставляти товар та мати успіх в своїх справах
 — поради старого Кур'єра» в якій описано проблему, але математичний апарат для її розв'язання не застосовується.
Натомість, в ній запропоновано приклади маршрутів для деяких регіонів Німеччини та Швейцарії. 
Раннім варіантом задачі є гра «Ікосіан» Вільяма Гамільтона XIX століття, яка полягала в тому,
щоб знайти маршрути на графі з 20 вузлами. Невдовзі з'явилась відома зараз назва задача мандруючого продавця,
яку запропонував Гаслер Вітні з Принстонського Університету.
В 1950-ті та 1960-ті роки задача комівояжера привернула увагу науковців в США та Європі. Важливий внесок в дослідження
задачі належить Джорджу Данцігу, Делберту Рею Фалкерсону та Селмеру Джонсону, котрі в 1954 році в інституті RAND Corportation
сформулювали задачу у вигляді задачі дискретної оптимізації та розробили метод відсікаючої площини для її розв'язання.
Використовуючи новий метод вони обчислили шлях для окремого набору вузлів (екземпляру проблеми) з 49 міст та довели,
що не існує коротшого шляху. В 1960-ті та 1970-ті роки численні групи дослідників вивчали задачу з точки зору математики
та її застосування, наприклад, в інформатиці, економіці, хімії та біології. 

\section{Практичне застосування TSP}
Задача комівояжера може застосовуватись для широкого спектра задач, в основі яких лежить проходження певного об'єкту
через множину пунктів так, щоб закінчення шляху збігалося з початком.
Прикладами задач в яких доцільне застосування даного методу є: 
\begin{enumerate}
    \item В ході планування робіт для кур'єра з доставки товарів. Планування маршруту проходження кур'єром пунктів доставки товару.
    \item В ході планування робіт для однієї машини. Планування маршруту від пункту початку(автостоянки) до пункту з запасами(складом)
    та подальше проходження всіх пунктів з потребами, з поверненням машини в кінці зміни на місце початку робіт.
    \item Також розв\textquotesingleязки задачі Комівояжера для 100 міст застосовують для
    синтезу петльових та непетльових антенних вібраторів формату 10x10 за допомогою мурашиного алгоритму оптимізації.
\end{enumerate}